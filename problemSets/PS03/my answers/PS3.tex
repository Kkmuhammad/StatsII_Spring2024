\documentclass[12pt,letterpaper]{article}
\usepackage{graphicx,textcomp}
\usepackage{natbib}
\usepackage{setspace}
\usepackage{fullpage}
\usepackage{color}
\usepackage[reqno]{amsmath}
\usepackage{amsthm}
\usepackage{fancyvrb}
\usepackage{amssymb,enumerate}
\usepackage[all]{xy}
\usepackage{endnotes}
\usepackage{lscape}
\newtheorem{com}{Comment}
\usepackage{float}
\usepackage{hyperref}
\newtheorem{lem} {Lemma}
\newtheorem{prop}{Proposition}
\newtheorem{thm}{Theorem}
\newtheorem{defn}{Definition}
\newtheorem{cor}{Corollary}
\newtheorem{obs}{Observation}
\usepackage[compact]{titlesec}
\usepackage{dcolumn}
\usepackage{tikz}
\usetikzlibrary{arrows}
\usepackage{multirow}
\usepackage{xcolor}
\newcolumntype{.}{D{.}{.}{-1}}
\newcolumntype{d}[1]{D{.}{.}{#1}}
\definecolor{light-gray}{gray}{0.65}
\usepackage{url}
\usepackage{listings}
\usepackage{color}

\definecolor{codegreen}{rgb}{0,0.6,0}
\definecolor{codegray}{rgb}{0.5,0.5,0.5}
\definecolor{codepurple}{rgb}{0.58,0,0.82}
\definecolor{backcolour}{rgb}{0.95,0.95,0.92}

\lstdefinestyle{mystyle}{
	backgroundcolor=\color{backcolour},   
	commentstyle=\color{codegreen},
	keywordstyle=\color{magenta},
	numberstyle=\tiny\color{codegray},
	stringstyle=\color{codepurple},
	basicstyle=\footnotesize,
	breakatwhitespace=false,         
	breaklines=true,                 
	captionpos=b,                    
	keepspaces=true,                 
	numbers=left,                    
	numbersep=5pt,                  
	showspaces=false,                
	showstringspaces=false,
	showtabs=false,                  
	tabsize=2
}
\lstset{style=mystyle}
\newcommand{\Sref}[1]{Section~\ref{#1}}
\newtheorem{hyp}{Hypothesis}

\title{Problem Set 3}
\date{Due: March 24, 2024}
\author{Applied Stats II}


\begin{document}
	\maketitle
	\section*{Instructions}
	\begin{itemize}
	\item Please show your work! You may lose points by simply writing in the answer. If the problem requires you to execute commands in \texttt{R}, please include the code you used to get your answers. Please also include the \texttt{.R} file that contains your code. If you are not sure if work needs to be shown for a particular problem, please ask.
\item Your homework should be submitted electronically on GitHub in \texttt{.pdf} form.
\item This problem set is due before 23:59 on Sunday March 24, 2024. No late assignments will be accepted.
	\end{itemize}

	\vspace{.25cm}
\section*{Question 1}
\vspace{.25cm}
\noindent We are interested in how governments' management of public resources impacts economic prosperity. Our data come from \href{https://www.researchgate.net/profile/Adam_Przeworski/publication/240357392_Classifying_Political_Regimes/links/0deec532194849aefa000000/Classifying-Political-Regimes.pdf}{Alvarez, Cheibub, Limongi, and Przeworski (1996)} and is labelled \texttt{gdpChange.csv} on GitHub. The dataset covers 135 countries observed between 1950 or the year of independence or the first year forwhich data on economic growth are available ("entry year"), and 1990 or the last year for which data on economic growth are available ("exit year"). The unit of analysis is a particular country during a particular year, for a total $>$ 3,500 observations. 

\begin{itemize}
	\item
	Response variable: 
	\begin{itemize}
		\item \texttt{GDPWdiff}: Difference in GDP between year $t$ and $t-1$. Possible categories include: "positive", "negative", or "no change"
	\end{itemize}
	\item
	Explanatory variables: 
	\begin{itemize}
		\item
		\texttt{REG}: 1=Democracy; 0=Non-Democracy
		\item
		\texttt{OIL}: 1=if the average ratio of fuel exports to total exports in 1984-86 exceeded 50\%; 0= otherwise
	\end{itemize}
	
\end{itemize}
\newpage
\noindent Please answer the following questions:

\begin{enumerate}
	\item Construct and interpret an unordered multinomial logit with \texttt{GDPWdiff} as the output and "no change" as the reference category, including the estimated cutoff points and coefficients.
	
	\textbf{		The R script for model fitting of unordered multinomial logit model is:}
	\lstinputlisting[language=R, firstline=42, lastline=58]{PS3.R}	 
	
	\vspace{.8cm}
	
		
		Coefficients:  \newline
				(Intercept)     				            gdp\_data\$OIL                   gdp\_data\$REG
		
		negative change                       3.805370     4.783968     1.379282 \newline
		positive change    					4.533759     4.576321     1.769007 \newline
		
		Std. Errors: \newline                
		(Intercept)                                    gdp\_data\$OIL             gdp\_data\$REG
		
		negative change             0.2706832     6.885366    0.7686958 \newline
		positive change              0.2692006     6.885097    0.7670366 \newline
		
		Residual Deviance: 4678.77 \newline
		AIC: 4690.77 
		
		The log odds when there is no change in both the ordinal variables is 3.8 while when the first variable OIL is changed to negative the log odds become 4.79 and when REG is changed to negative keeping the other variables constant, becomes 1.38.
\newpage	
	\item Construct and interpret an ordered multinomial logit with \texttt{GDPWdiff} as the outcome variable, including the estimated cutoff points and coefficients.
	
	\textbf{		The R script for model fitting of ordered multinomial logit model is:}
	\lstinputlisting[language=R, firstline=61, lastline=69]{PS3.R}
	
	Coefficients: \newline              
	Value 					Std. Error 						t value \newline
	gdp\_data\$OIL -0.1987    0.11572  -1.717 \newline
	gdp\_data\$REG  0.3985    0.07518   5.300\newline
	
	Intercepts:  \newline
	Value    Std. Error t value \newline
	negative change|no change  -0.7312   0.0476   -15.3597\newline
	no change|positive change  -0.7105   0.0475   -14.9554\newline
	
	Residual Deviance: 4687.689 \newline
	AIC: 4695.689  
	
	The log odds in ordered multinomial logit model of OIL when the other variable is constant is -0.19 while the log odds of REG when OIL is constant is 0.39.  This is because of chnage in category of GDP from negative to positive.
\end{enumerate}

\section*{Question 2} 
\vspace{.25cm}

\noindent Consider the data set \texttt{MexicoMuniData.csv}, which includes municipal-level information from Mexico. The outcome of interest is the number of times the winning PAN presidential candidate in 2006 (\texttt{PAN.visits.06}) visited a district leading up to the 2009 federal elections, which is a count. Our main predictor of interest is whether the district was highly contested, or whether it was not (the PAN or their opponents have electoral security) in the previous federal elections during 2000 (\texttt{competitive.district}), which is binary (1=close/swing district, 0="safe seat"). We also include \texttt{marginality.06} (a measure of poverty) and \texttt{PAN.governor.06} (a dummy for whether the state has a PAN-affiliated governor) as additional control variables. 

\begin{enumerate}
	\item [(a)]
	Run a Poisson regression because the outcome is a count variable. Is there evidence that PAN presidential candidates visit swing districts more? Provide a test statistic and p-value.

	\textbf{		The R script for poisson regression is:}
	\lstinputlisting[language=R, firstline=76, lastline=91]{PS3.R}
	
	Coefficients: \newline                                     
	Estimate Std. Error z value Pr(>|z|)   \newline 
	(Intercept)                     -3.81023    0.22209 -17.156   <2e-16 ***\newline
	 mexico\_elections\$competitive.district -0.08135    0.17069  -0.477   0.6336   \newline mexico\_elections\$marginality.06       -2.08014    0.11734 -17.728   <2e-16 *** \newline
	 mexico\_elections\$PAN.governor.06      -0.31158    0.16673  -1.869   0.0617 \newline 
	 ---Signif. codes:  0 ‘***’ 0.001 ‘**’ 0.01 ‘*’ 0.05 ‘.’ 0.1 ‘ ’ 1 \newline
	 (Dispersion parameter for poisson family taken to be 1)   \newline
	  Null deviance: 1473.87  on 2406  degrees of freedom\newline
	  Residual deviance:  991.25  on 2403  degrees of freedom
	
	The p value of visit to competitive districts is 0.6336 which is higher than the benchmark 0.05. As the null hypothesis represents status quo, therefore we cannot accept the alternative hypothesis of more visits to competitive districts. 
	
	\item [(b)]
	Interpret the \texttt{marginality.06} and \texttt{PAN.governor.06} coefficients.
	
	The coefficient of marginality is -2.08 which means that when there is 2 units poverty the presidential candidate visited the district. while the coefficient of PAN governor is -0.31 which predicts that the number of visits by a presential candidate is -0.31.
	
	\newpage
	
	\item [(c)]
	Provide the estimated mean number of visits from the winning PAN presidential candidate for a hypothetical district that was competitive (\texttt{competitive.district}=1), had an average poverty level (\texttt{marginality.06} = 0), and a PAN governor (\texttt{PAN.governor.06}=1).
	
	\textbf{		The R script for poisson regression is:}
	\lstinputlisting[language=R, firstline=93, lastline=107]{PS3.R}
	
	The estimated mean number of visits from the winning PAN presidential candidate for a hypothetical district is 0.91.
	
\end{enumerate}

\end{document}
